\section{Analyse}

Problème: Affecter des températures (une maximale et l’autre minimale) 
pour chacun des jours des années comprises entre 1901 et 2099,
ensuite transférer les données générées dans un fichier texte
dans la structure sera lisible et claire par un éditeur de texte.

Ensuite, vient l’étape de la lecture, qui consiste a extraire les données
 stockées dans le fichier creé et les stocker dans une 
 structure de données compatible, pour faciliter la lecture des températures
et ainsi mieux exploiter les données du fichier, avec des fonctions d’interrogations.

En ce qui concerne l'interrogation on peut voir les données lues à partir du fichier, 
comme une table dans une base de données ayant les colonnes suivantes : 

\begin{itemize}
  \item \lstinline$jour$ : Le jour.
  \item \lstinline$mois$ : Le mois.
  \item \lstinline$an$ : L'année.
  \item \lstinline$min$ : La température minimum du jour.
  \item \lstinline$max$ : La tempréture maximum du jour.
\end{itemize}

En plus de ces informations qu'on peut onbtenir directement de la structure de donnée, on peut
ajouter d'autres qui sont le résultat d'un calcule effectuer sur un an, mois ou jour donnée.

\begin{itemize}
  \item \lstinline$moy(jour)$ : La température moyenne du jour. 
  
  \item \lstinline$min(mois)$ : La température minimum du mois.
  \item \lstinline$max(mois)$ : La température maximum du mois.
  \item \lstinline$moy(mois)$ : La température moyenne du mois.
  
  \item \lstinline$min(an)$ : La température minimum de l'année.
  \item \lstinline$max(an)$ : La température maximum de l'année.
  \item \lstinline$moy(an)$ : La température moyenne de l'année.
\end{itemize}

Après qu l'utilisateur a entré les colonnes qu'il veut afficher,
il doit saisir les conditions qui doivent etre vrai pour les lignes
résultats. L'exemple suivant montre le déroulement du programme de lecture des données.

\begin{lstlisting}
Entrez le nom du fichier : temps.txt
COLONNES : 
    an mois jour min max 
CONDITIONS : 
    jour est mardi et an = 2001 et min < 10
ou an = 1901 et mois = 1 et jour = 1 ;
1901 1 1 1 9 
2001 1 2 -7 9 
2001 1 9 0 8 
2001 1 16 -4 8 
2001 1 23 2 8 
...
\end{lstlisting}  

Afin de disposer d’un maximum d’information pour l’analyse et la résolution du problème posé,
il est nécessaire d’avoir des connaissances au sujet de la structure du calendrier et des saisons.

\begin{itemize}
\item Une année normale est composée de 365 jours, une année bissextile est composée de 366 jours,
\item Une année bissextile est toujours divisible par 4.
\item Une année est composée de 12 mois.
\item Chaque mois est composé d’un nombre de jours variant entre 28 et 31; les mois 
1,3,5,7,8,10,12 ont 31 jours, les mois 4,6,9,11 ont 30 jours et le mois 2 a 29 jours 
dans le cas d’une année bissextile et 28 jours dans le cas normal.


\item Un jour sera caractérisé par deux températures (la premiere représente la valeur minimale
 enregistrée pendant la journée,
 la seconde est celle de la valeur maximale).
 
\item Une année est composée de 4 trimestres chaque trimestre est une saison dont la 
longueur est approximative à 90 jours.

\item Lors de la generation des temperatures et pour rester dans le domaine de la 
logique (ne pas avoir des temperatures avec le temps d’un mois d’une saison quelconque) 
(ex : température max = 45 au mois de décembre.) chaque mois aura un interval de temperatures
 bien défini et bien compatible), en plus de la condition (min < max).

\item Pour l’affectation des intervals de temperatures possibles pour chaque mois, 
on vas commencer par affecter a chaque mois un nombre t,
 ainsi le min sera dans l’interval [t-12, t-2] et le max dans l’interval [t+2, t+12].
  Ainsi on aura verifier les deux conditions de validité des données générées
   (des temperatures compatibles avec chaque saison et logiques(min < max)).
\end{itemize}

\begin{description}
\item[Exemple 1:] Les temperatures min et max du jour 2000/01/12.
pour le mois 1 on aurais défini t = 5.
ainsi: min sera dans [-7, 3] min = 2.
       max sera dans [7, 17] max = 15.
 
\item[Exemple 2:] Les temperatures min et max du jour 1982/07/09.
pour le mois 1 on aurais défini t = 24.
ainsi: min sera dans [12, 22] min = 15.
       max sera dans [26, 36] max = 33.

\item[Exemple 3:] Les temperatures min et max du jour 1952/04/23.
pour le mois 1 on aurais défini t = 11.
ainsi: min sera dans [-1, 9] min = 8.
       max sera dans [13, 23] max =21.
\end{description}



         